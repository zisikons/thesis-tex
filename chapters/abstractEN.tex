\chapter*{Abstract}

The objective of this thesis is the problem of nonlinear, time invariant, continuous time system identification in a certain area of interest. The difficulty of the aftermentioned problem lies in the satisfaction of the \textit{Persistency of Excitation} condition. While there are many studies on nonlinear system identification, only a few consider the a priori satisfaction of the \textit{Persistency of Excitation} condition, which is the main goal of this study.

The results of \textit{Kurdila}, \textit{Narcowich} and \textit{Ward} in~\cite{kurdila1995persistency} provide the necessary conditions for the satisfaction of the \textit{Persistency of Excitation} when using \textit{Radial Basis Function (RBF) Approximants}. This work provides the fundamental theoretical background for the development of the proposed identification scheme.

The basic components of this study consist of the \textit{RBF Neural Networks}, which are an ideal mathematical model for universal nonlinear function approximation due to their approximation capabilities~\cite{park1991universal}, as well as the \textit{Prescribed Performance Control} methodology~\cite{bechlioulis2008robust} which enables trajectory tracking even under complete lack of knowledge on the controlled system.  

Based on the after mentioned results, we present an identification scheme capable of \textit{a priori} guaranteeing the satisfaction of the \textit{Persistency of Excitation} condition, thus achieving identification of the underlying nonlinear dynamics of the given system in a predefined area of interest.

Finally, using mathematical arguments such as \textit{Lyapunov} stability theory as well as computer simulations of real world systems, we provide satisfactory results to demonstrate the effectiveness of the proposed scheme. 
