\chapter*{Σύνοψη}
Στην παρούσα διπλωματική εργασία θα ασχοληθούμε με το πρόβλημα της αναγνώρισης μη-γραμμικών χρονοαμετάβλητων δυναμικών συστημάτων συνεχούς χρόνου, σε ένα υποσύνολο του χώρου λειτουργίας τους. Η δυσκολία του εν λόγω προβλήματος οφείλεται σε μεγάλο βαθμό στο πρόβλημα της ικανοποίησης της συνθήκης της \textit{επιμένουσας διέγερσης}. Ενώ πολλές δημοσιευμένες εργασίες έχουν ασχοληθεί με το πρόβλημα της αναγνώρισης μη-γραμμικών συστημάτων, ελάχιστες είναι αυτές που ασχολούνται με την ικανοποίηση της συνθήκης της \textit{επιμένουσας διέγερσης} εκ των προτέρων, το οποίο αποτελεί τον στόχο της παρούσας μελέτης.

Βάση αυτής της εργασίας αποτελεί η έρευνα των \textit{Kurdila}, \textit{Narcowich} και \textit{Ward} \cite{kurdila1995persistency}, η οποία παρουσιάζει κάποιες απαραίτητες προϋποθέσεις για την ικανοποίηση της συνθήκης \textit{επιμένουσας διέγερσης} για την κλάση των μαθηματικών μοντέλων \textit{RBF}.

Απαραίτητα δομικά στοιχεία αυτής της εργασίας είναι αφενός τα \textit{τεχνητά νευρωνικά δίκτυα RBF}, τα οποία λόγω των προσεγγιστικών τους ιδιοτήτων \cite{park1991universal} αποτελούν ένα άριστο μαθηματικό μοντέλο για το πρόβλημα της τοπικής αναγνώρισης συναρτήσεων, και αφετέρου ο \textit{έλεγχος προδιαγεγραμμένης απόκρισης} \cite{bechlioulis2008robust} ο οποίος μας επιτρέπει την παρακολούθηση μιας επιθυμητής τροχιάς ακόμα και υπό την πλήρη έλλειψη γνώσεων για το ελεγχόμενο σύστημα.

Συνδυάζοντας τα παραπάνω εργαλεία, παρουσιάζουμε ένα σχήμα αναγνώρισης το οποίο εξασφαλίζει την ικανοποίηση της συνθήκης της επιμένουσας διέγερσης και κατά συνέπεια την επιτυχή αναγνώριση της δυναμικής του άγνωστου συστήματος σε μια προκαθορισμένη περιοχή του χώρου λειτουργίας του δοθέντος συστήματος. 

Τέλος, τόσο με την χρήση μαθηματικών επιχειρημάτων όπως οι συναρτήσεις \textit{Lyapunov} αλλά όσο και με την χρήση προσομοιώσεων, αποδεικνύεται η ορθότητα της προαναφερθείσας μεθοδολογίας.



%Στα πλαίσια της εργασίας θα ασχοληθούμε με αρχικά με τον μαθηματικό σχεδιασμό ενός αλγορίθμου εκτίμησης παραμέτρων. Το εργαλείο που θα χρησιμοποιηθεί τόσο για την εξαγωγή των σχέσεων, όσο και για την ανάλυση επίδοσης του αλγορίθμου είναι η μεθοδολογία σύνθεσης \textit{Lyapunov}. Στην συνέχεια θα προχωρήσουμε στην επικύρωση των αποτελεσμάτων μέσω προσομοιώσεων για πραγματικά συστήματα όπως ο ταλαντωτής \textit{Van Der Pol}, και τέλος θα μελετηθούν τεχνικές επικύρωσης αποτελεσμάτων οι οποίες είναι απαραίτητες για την διαδικασία αναγνώρισης συστημάτων στην πράξη.

%Σε όλη την έκταση της διπλωματικής εργασίας θα συζητηθούν οι επιπτώσεις των σχεδιαστικών επιλογών, οι πιθανές εναλλακτικές σε κάθε περίπτωση, καθώς και τα πλεονεκτήματα και οι πιθανές δυσκολίες εφαρμογής του αλγορίθμου υπό πραγματικές συνθήκες.