\chapter*{Σύνοψη}
\subsubsection{Ιστορικό - να εδώ}
Από την πρώτη της εμφάνισης του πάνω στη γη, ο άνθρωπος κυριαρχείται από μια εναγώνια προσπάθεια κατάκτησης γνώσεων. Ορίζοντας ώς σύστημα κάθε αντικείμενο ή ομάδα αντικειμένων τις ιδιότητες τον οποίων θέλουμε να μελετήσουμε, με τον όρο \textit{αναγνώριση συστημάτων} αναφερόμαστε στην διαδικασία εξαγωγής ενός μαθηματικού μοντέλου ενός πραγματικού συστήματος με βάση πειραματικά δεδομένα. 

Το πρόβλημα της αναγνώρισης συστημάτων απασχολεί την επιστημονική κοινότητα για πάνω από μισό αιώνα. Το βασικό κίνητρο είναι πως ένα "καλό" μοντέλο του πραγματικού συστήματος είναι απαραίτητο για μια πληθώρα εφαρμογών όπως αυτές του αυτομάτου ελέγχου, της προσομοίωσης, της πρόβλεψης, της εκτίμησης καταστάσεων καθώς και της ανίχνευσης σφαλμάτων. Έτσι λοιπόν ο σχεδιασμός κατάλληλων πειραμάτων, η επιλογή μαθηματικών μοντέλων καθώς και η ανάπτυξη αλγορίθμων εκτίμησης παραμέτρων αποτελούν  μέχρι και σήμερα πεδίο διαρκούς έρευνας.

Στην παρούσα διπλωματική εργασία θα ασχοληθούμε με το πρόβλημα της αναγνώρισης μη-γραμμικών χρονοαμετάβλητων δυναμικών συστημάτων συνεχούς χρόνου, σε ένα υποσύνολο του χώρου λειτουργίας τους. Πιο συγκεκριμένα, σκοπός είναι ο έλεγχος του άγνωστου μη γραμμικού συστήματος με σκοπό την ταυτόχρονη εκτίμηση παραμέτρων ενός μοντέλου αυτού. 

Αναλυτικότερα, στα πλαίσια της εργασίας αυτής προτείνεται ένα παραμετροποιημένο μαθηματικό μοντέλο με την ικανότητα προσέγγισης του πραγματικού συστήματος σε οποιοδήποτε βαθμό ακρίβειας, καθώς και ένας αλγόριθμος για την ορθή εκτίμηση παραμέτρων του μοντέλου αυτού. Στην συνέχεια, μέσω τόσο πειραμάτων, όσο και μαθηματικών επιχειρημάτων αποδεικνύεται η ικανότητα αυτής της μεθόδου να αναγνωρίσει επαρκώς το άγνωστο σύστημα. Τέλος, εξετάζονται τρόποι επικύρωσης των αποτελεσμάτων υπό πραγματικές συνθήκες το οποίο αποτελεί ένα απαραίτητο βήμα για την αναγνώριση συστημάτων σε πρακτικές εφαρμογές.



