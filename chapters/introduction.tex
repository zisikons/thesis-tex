\chapter{Εισαγωγή}

Από την πρώτη της εμφάνισης του πάνω στη γη, ο άνθρωπος κυριαρχείται από μια εναγώνια προσπάθεια κατάκτησης γνώσεων. Ορίζοντας ώς σύστημα κάθε αντικείμενο ή ομάδα αντικειμένων τις ιδιότητες τον οποίων θέλουμε να μελετήσουμε, με τον όρο \textit{αναγνώριση συστημάτων} αναφερόμαστε στην διαδικασία εξαγωγής ενός μαθηματικού μοντέλου του πραγματικού συστήματος με βάση πειραματικά δεδομένα. 

\textbf{Η εισαγωγική παράγραφος μπορεί να αλλάξει} \\
Το πρόβλημα της αναγνώρισης συστημάτων απασχολεί την επιστημονική κοινότητα για πάνω από μισό αιώνα. Το βασικό κίνητρο είναι πως ένα "καλό" μοντέλο του πραγματικού συστήματος είναι απαραίτητο για μια πληθώρα εφαρμογών %όπως αυτές του αυτομάτου ελέγχου, της προσομοίωσης, της πρόβλεψης, της εκτίμησης καταστάσεων καθώς και της ανίχνευσης σφαλμάτων. 
Έτσι λοιπόν ο σχεδιασμός κατάλληλων πειραμάτων, η επιλογή μαθηματικών μοντέλων καθώς και η ανάπτυξη αλγορίθμων εκτίμησης παραμέτρων αποτελούν  μέχρι και σήμερα πεδίο διαρκούς έρευνας.

\section{Εισαγωγικές Έννοιες}
Σκοπός αυτού του κεφαλαίου είναι η παρουσίαση των βασικών εννοιών της θεωρίας \textit{Αναγνώρισης Συστημάτων}. Με αυτό τον τρόπο ελπίζουμε αφενός να γίνει πλήρως κατανοητός ο σκοπός της παρούσης εργασίας και αφετέρου να αποσαφηνισθoύν οι διαφορές με άλλες κλασσικές μεθόδους αναγνώρισης συστημάτων.

\subsection{Μαθηματικά Μοντέλα}
Όπως είπαμε, το αποτέλεσμα της αναγνώρισης συστημάτων στα πλαίσια που την μελετάμε ονομάζεται \textit{μοντέλο}. Υπάρχουν πολλές κατηγορίες μοντέλων όπως τα λεκτικά, μαθηματικά, φυσικά και άλλα, ωστόσο στα πλαίσια αυτής της εργασίας θα εργαστούμε με τα μαθηματικά μοντέλα συστημάτων.

%A model structure is a mathematical relationship between input and output variables that contains unknown parameters. Examples of model structures are transfer functions with adjustable poles and zeros, state space equations with unknown system matrices, and nonlinear parameterized functions.

Στην περίπτωση μας λοιπόν, ένα μοντέλο είναι μια μαθηματική σχέση μεταξύ μεταβλητών εισόδου και εξόδου που περιέχει άγνωστες παραμέτρους. Παραδείγματα μαθηματικών μοντέλων αποτελούν οι συναρτήσεις μεταφοράς με μεταβλητά μηδενικά και πόλους, οι εξισώσεις κατάστασης με άγνωστους πίνακες καταστάσεων καθώς και οι παραμετροποιημένες μη-γραμμικές συναρτήσεις.

Για παράδειγμα, η παρακάτω διαφορική εξίσωση αποτελεί ένα απλό μαθηματικό μοντέλο:

\begin{equation}
	\dot{x}(t) = -a x(t) + b u(t)
	%\dot{x} (x) = 5
\end{equation}

%όπου οι μεταβλητές $a$ και $b$ είναι οι ελευθέροι παράμετροι.

%y(k)+ay(k−1)=bu(k)
%where a and b are adjustable parameters.

%The system identification process requires that you choose a model structure and apply the estimation methods to determine the numerical values of the model parameters.

%Η πολυπλοκότητα του επιλεγμένου μοντέλου θα πρέπει να εξυπηρετεί της απαιτήσεις της εκάστοτε εφαρμογής αναγνώρισης.

Add: Βασικά χαρακτηρηστικά των μοντέλων (ελεύθεροι παράμετροι, etc)
Ένα μοντέλο θα πρέπει να είναι συμπαγές, πλήρες 

Λόγω της πληθώρας και της ιδιαιτερότητας των συστημάτων, έχουν προταθεί διάφορα μαθηματικά μοντέλα με χαρακτηριστικά που εξαρτώνται από τις ιδιότητες του προς μελέτη συστήματος. Διακρίνουμε τους παρακάτω τύπους μαθηματικών μοντέλων:

\begin{itemize}
	\item{ \textbf{Ντετερμινιστικά - Στοχαστικά} }
\end{itemize}

\subsection{Αλγόριθμοι εκτίμησης παραμέτρων}


\section{Εφαρμογές της Αναγνώρισης Συστημάτων}

\section{Διαδικασία Αναγνώρισης Συστημάτων}

\section{Σημαντική Βιβλιογραφία}

\section{Δομή της Διπλωματικής Εργασίας}

\subsection{(τίτλος υποενότητας 1.1.1)}


