\chapter{Εισαγωγή}

Από την πρώτη της εμφάνισης του πάνω στη γη, ο άνθρωπος κυριαρχείται από μια εναγώνια προσπάθεια κατάκτησης γνώσεων. Ορίζοντας ώς σύστημα κάθε αντικείμενο ή ομάδα αντικειμένων τις ιδιότητες τον οποίων θέλουμε να μελετήσουμε, με τον όρο \textit{αναγνώριση συστημάτων} αναφερόμαστε στην διαδικασία εξαγωγής ενός μαθηματικού μοντέλου του πραγματικού συστήματος με βάση πειραματικά δεδομένα. 

%\textbf{Η εισαγωγική παράγραφος μπορεί να αλλάξει} \\
Το πρόβλημα της αναγνώρισης συστημάτων απασχολεί την επιστημονική κοινότητα για πάνω από μισό αιώνα. Το βασικό κίνητρο είναι πως ένα "καλό" μοντέλο του πραγματικού συστήματος είναι απαραίτητο για μια πληθώρα εφαρμογών. Έτσι λοιπόν ο σχεδιασμός κατάλληλων πειραμάτων, η επιλογή μαθηματικών μοντέλων καθώς και η ανάπτυξη αλγορίθμων εκτίμησης παραμέτρων αποτελούν  μέχρι και σήμερα πεδίο διαρκούς έρευνας.

\section{Εισαγωγικές Έννοιες}
Σκοπός αυτού του κεφαλαίου είναι η παρουσίαση των βασικών εννοιών της θεωρίας \textit{Αναγνώρισης Συστημάτων}. Με αυτό τον τρόπο ελπίζουμε αφενός να γίνει πλήρως κατανοητός ο σκοπός της παρούσης εργασίας και αφετέρου να αποσαφηνισθoύν οι διαφορές με άλλες κλασσικές μεθόδους αναγνώρισης συστημάτων.

\subsection{Μαθηματικά Μοντέλα}
Όπως είπαμε, το αποτέλεσμα της αναγνώρισης συστημάτων στα πλαίσια που την μελετάμε ονομάζεται \textit{μοντέλο}. Υπάρχουν πολλές κατηγορίες μοντέλων όπως τα λεκτικά, μαθηματικά, φυσικά και άλλα, ωστόσο στα πλαίσια αυτής της εργασίας θα εργαστούμε με τα μαθηματικά μοντέλα συστημάτων.

Στην περίπτωση μας λοιπόν, ένα μοντέλο είναι μια μαθηματική σχέση μεταξύ μεταβλητών εισόδου και εξόδου που περιέχει ελεύθερες παραμέτρους. Παραδείγματα μαθηματικών μοντέλων αποτελούν οι συναρτήσεις μεταφοράς με μεταβλητά μηδενικά και πόλους, οι εξισώσεις κατάστασης με άγνωστους πίνακες καταστάσεων καθώς και οι παραμετροποιημένες μη-γραμμικές συναρτήσεις.

Για παράδειγμα, η παρακάτω διαφορική εξίσωση αποτελεί ένα απλό μαθηματικό μοντέλο:
\begin{equation*}
	\dot{x}(t) = -a x(t) + b u(t)
\end{equation*}
όπου οι μεταβλητές $a$ και $b$ είναι οι ελευθέροι παράμετροι του μοντέλου. Η πολυπλοκότητα του επιλεγμένου μοντέλου θα πρέπει να εξυπηρετεί της απαιτήσεις της εκάστοτε εφαρμογής αναγνώρισης.

Λόγω της πληθώρας και της ιδιαιτερότητας των συστημάτων, έχουν προταθεί διάφορα μαθηματικά μοντέλα με χαρακτηριστικά που εξαρτώνται από τις ιδιότητες του προς μελέτη συστήματος. Παρακάτω αναφέρουμε κάποιες βασικές υποκατηγορίες μαθηματικών μοντέλων.

\begin{itemize}
	\item{ 
	\textbf{Ντετερμινιστικά - Στοχαστικά.} Ένα μοντέλο ονομάζεται ντετερμινιστικό, αν περιγράφεται από μια πλήρως προσδιορισμένη σχέση μεταξύ των μεταβλητών του. Αντιθέτως θα λέγεται στοχαστικό, αν εκφράζεται μέσω πιθανοθεωρίας.
	}
	
	\item{ 
	\textbf{Στατικά - Δυναμικά.} Εάν η σχέση (μοντέλο) που συνδέει τις μεταβλητές ενός συστήματος δεν εξαρτάται από παρελθοντικές τιμές των μεταβλητών θα λέμε ότι το σύστημα, άρα και το μοντέλο, είναι στατικό. Στην αντίθετη περίπτωση θα λέγεται δυναμικό. Ένα παράδειγμα στατικού συστήματος είναι αυτό που περιγράφεται από αλγεβρικές εξισώσεις, ενώ συνήθως τα δυναμικά συστήματα - μοντέλα περιγράφονται από διαφορικές εξισώσεις ή εξισώσεις διαφορών.
	}
	
	\item{ 
	\textbf{Συνεχούς - Διακριτού Χρόνου.} Η ιδιότητα αυτή ορίζει τον τρόπο με τον οποίο η μεταβλητή του χρόνου επιδρά στο μοντέλο. Διακριτού χρόνου ονομάζονται τα μοντέλα τα οποία εκφράζουν την σχέση που συνδέει τις μεταβλητές του συστήματος σε διακριτές χρονικές στιγμές. Eν αντιθέσει, τα συστήματα στα οποία ο χρόνος είναι συνεχής μεταβλητή ονομάζονται συστήματα συνεχούς χρόνου. Τα μοντέλα διακριτού χρόνου περιγράφονται συνήθως από εξισώσεις διαφορών ενώ τα συστήματα συνεχούς χρόνου από διαφορικές εξισώσεις.
	}
	
	\item{ 
	\textbf{Χρονομεταβλητά - Χρονοαμετάβλητα.} Χρονομεταβλητά ονομάζονται τα συστήματα (και τα μοντέλα συστημάτων) στα οποία οι εξισώσεις που περιγράφουν την λειτουργιά του συστήματος έχουν άμεση εξάρτηση από τον χρόνο. Κατά συνέπεια, σε αυτά τα συστήματα είναι πιθανό η ίδια είσοδος σε διαφορετικές χρονικές στιγμές να οδηγήσει σε διαφορετική απόκριση του συστήματος. Έν αντιθέσει, χρονοαμετάβλητα είναι τα συστήματα στα οποία η εξάρτηση με τον χρόνο εκφράζεται μόνο έμμεσα μέσω των εσωτερικών καταστάσεων ή της συνάρτησης εισόδου του συστήματος. 
	
	
%	Χρονομεταβλητά ονομάζονται τα συστήματα (και τα μοντέλα συστημάτων) στα οποία ο χρόνος έχει άμεση επίδραση. Με άλλα λόγια είναι συστήματα στα οποία κάποιες συγκεκριμένες ποσότητες που επηρεάζουν την συμπεριφορά του συστήματος έχουν άμεση συσχέτιση με τον χρόνο, και ως εκ τούτου η ίδια είσοδος σε διαφορετικές χρονικές στιγμές ενδέχεται να οδηγήσει σε διαφορετικές αποκρίσεις. Αντίθετα στα χρονοαμετάβλητα συστήματα, οι εξισώσεις που περιγράφουν την λειτουργιά του συστήματος δεν έχουν άμεση εξάρτηση από τον χρόνο.
	
	}
\end{itemize}

Τα συστήματα που θα μελετήσουμε σε αυτή την εργασία περιγράφονται από μη-γραμμικές διαφορικές εξισώσεις όπου οι μη-γραμμικές συναρτήσεις που διέπουν την λειτουργία τους εξαρτώνται μόνο από τις καταστάσεις και την είσοδο ελέγχου. Με βάση την παραπάνω κατηγοριοποίηση λοιπόν, τα συστήματα (και τα αντίστοιχα μοντέλα) που μελετάμε είναι μη-γραμμικά, χρονοαμετάβλητα δυναμικά συστήματα συνεχούς χρόνου.

\pagebreak
\subsection{Εκτίμηση Παραμέτρων}
Στην αναγνώριση συστημάτων, αφού επιλέξουμε ένα μαθηματικό μοντέλο με την ικανότητα να περιγράψει επαρκώς την λειτουργία του συστήματος, το επόμενο στάδιο είναι ο προσδιορισμός των ελεύθερων του παραμέτρων. Η διαδικασία αυτή στην βιβλιογραφία ονομάζεται \textit{εκτίμηση παραμέτρων} (parameter estimation).

%H διαδικασία αυτή περιλαμβάνει τον σχεδιασμό ενός πειράματος με σκοπό την συλλογή δεδομένων για το σύστημα που μελετάται, και στην συνέχεια την χρήση τους από κάποιο \textit{αλγόριθμο εκτίμησης παραμέτρων} με σκοπό 

Σε αυτό το στάδιο, είναι απαραίτητος ο σχεδιασμός ενός πειράματος με σκοπό την συλλογή δεδομένων για το σύστημα που μελετάται. Στην συνέχεια, τα δεδομένα χρησιμοποιούνται από κάποιον αλγόριθμο με σκοπό τον προσδιορισμό των ελεύθερων παραμέτρων του μοντέλου που έχει επιλεχθεί. Η αποτελεσματικότητα του αλγορίθμου εκτίμησης παραμέτρων είναι άμεση συνάρτηση των δεδομένων που θα χρησιμοποιηθούν, συνεπώς απαιτείται προσοχή κατά τον σχεδιασμό του πειράματος συλλογής δεδομένων.

Διακρίνονται δύο μεγάλες οικογένειες αλγορίθμων εκτίμησης παραμέτρων:

\begin{itemize}
	\item{\textbf{Offline:} Οι \textit{offline} αλγόριθμοι απαιτούν την εκ των προτέρων συλλογή δεδομένων για το διαθέσιμο σύστημα. Στην συνέχεια τα δεδομένα αυτά επεξεργάζονται από κάποιον αλγόριθμο εκτίμησης παραμέτρων με σκοπό την προσαρμογή ενός υποψηφίου μοντέλου του συστήματος. Το μεγάλο πλεονέκτημα των \textit{offline} αλγορίθμων είναι το γεγονός ότι δεν υπάρχει φραγμός ως προς την υπολογιστική τους πολυπλοκότητα, συνεπώς μπορούν να χρησιμοποιηθούν σύνθετοι αλγόριθμοι βελτιστοποίησης που καθιστούν εφικτή την προσαρμογή ακόμα και πολύ σύνθετων μοντέλων όπως τα \textit{πολυεπίπεδα νευρωνικά δίκτυα} (Deep Neural Networks). Ένα κλασσικό παράδειγμα τέτοιου αλγορίθμου είναι η \textit{Μέθοδος των Ελαχίστων Τετραγώνων}.
	}
		
	\item{\textbf{Online:} Αντίθετα με τους \textit{offline} αλγορίθμους οι \textit{online} αλγόριθμοι πραγματοποιούν εκτίμηση παραμέτρων σε πραγματικό χρόνο χρησιμοποιώντας τα δεδομένα που συλλέγονται κατά την διάρκεια λειτουργίας του πραγματικού συστήματος. Το γεγονός αυτό επιβάλει σε αυτούς τους αλγορίθμους να είναι υπολογιστικά απλοί, καθώς οι υπολογισμοί δεν μπορούν να διαρκούν περισσότερο από τον κύκλο λειτουργίας του συστήματος. 
		
	Ωστόσο, το πλεονέκτημα που προσφέρουν είναι ότι το μοντέλο είναι διαθέσιμο κατά την διάρκεια λειτουργίας του συστήματος, ιδιότητα που τους καθιστά ιδιαίτερα χρήσιμους σε εφαρμογές όπως ο \textit{προσαρμοστικός έλεγχος} (adaptive control) και η \textit{διάγνωση βλαβών} (fault detection). Ένας κλασσικός \textit{online} αλγόριθμος εκτίμησης παραμέτρων είναι η \textit{Αναδρομική Μέθοδος Ελαχίστων Τετραγώνων}.
	}
\end{itemize}

Για τους σκοπούς της παρούσας εργασίας η μέθοδος εκτίμησης που θα αναπτυχθεί πρόκειται για μια \textit{online} μέθοδο.

\section{Εφαρμογές της Αναγνώρισης Συστημάτων}
Σε αυτό το κεφάλαιο αναφέρουμε ενδεικτικά κάποιες από τις  τυπικές εφαρμογές της Αναγνώρισης Συστημάτων.

\begin{itemize}
	\item
	{
		\textbf{Πρόβλεψη.} Η ιδέα είναι ότι αν καταλήξουμε σε μια πολύ καλή μαθηματική περιγραφή του συστήματος, τότε μπορούμε να το επιλύσουμε για μελλοντικές χρονικές στιγμές, προβλέποντας με τον τρόπο αυτό την απόκριση του πραγματικού συστήματος σε κάποιο βάθος χρόνου που ονομάζεται χρονικός ορίζοντας (time horizοn). Εδώ αξίζει να σημειωθεί πως η συγκεκριμένη εφαρμογή απαιτεί μοντέλα πολύ υψηλής ακρίβειας όσο ο χρονικός ορίζοντας μεγαλώνει.
		
	}
	
	\item 
	{
		\textbf{Έλεγχος Συστημάτων.} Παρόλο που στην βιβλιογραφία υπάρχουν ελεγκτές που μπορούν να ελέγχουν συστήματα ακόμα και υπό την έλλειψη γνώσης του μοντέλου τους, η γνώση μιας μαθηματικής περιγραφής του συστήματος επιτρέπει την σχεδίαση πολύ πιο αποτελεσματικών ελεγκτών. Τα μοντέλα που απαιτούνται για τέτοιες εφαρμογές συνήθως είναι πιο απλά έτσι ώστε να είναι εύκολη η χρήση τους στον βρόγχο ελέγχου.
	}
	
	\item 
	{
		\textbf{Εκτίμηση Καταστάσεων.} Υπάρχουν περιπτώσεις, όπου δεν είναι διαθέσιμες προς μέτρηση όλες οι καταστάσεις ενός συστήματος, είτε λόγω κόστους, είτε λόγω έλλειψης αξιόπιστης μεθόδου μέτρησης τους. Σε αυτές τις περιπτώσεις, η ύπαρξη ενός μοντέλου του συστήματος μπορεί να οδηγήσει στην έμμεση μέτρηση ή διαφορετικά στην εκτίμηση των καταστάσεων. Η ποιότητα της εκτίμησης είναι άμεση συνάρτηση της ποιότητας του μοντέλου.
	}
	
	\item 
	{
		\textbf{Προσομοίωση.} Είναι η αριθμητική επίλυση του μοντέλου. Χρησιμοποιείται σε κάθε πρόβλημα σχεδίασης ως μέσο εκτίμησης της απόδοσης του αναπτυσσόμενου συστήματος, στην εκπαίδευση των χειριστών του συστήματος, αλλά και στην υποβοήθηση του ελέγχου καλής λειτουργίας και στην λήψη αποφάσεων.  
	}
	
	\item 
	{
		\textbf{Βελτιστοποίηση.} Όλα τα μαθηματικά εργαλεία που εγγυώνται την βέλτιστη λειτουργία ενός συστήματος στηρίζονται στην ύπαρξη ενός μοντέλου. Επειδή η λύση που προσδιορίζεται είναι κάθε φορά βέλτιστη για το μαθηματικό μοντέλο του συστήματος, και όχι το ίδιο το σύστημα, είναι προφανές πως και η εγκυρότητα της λύσης για το πραγματικό σύστημα είναι ανάλογη της ποιότητας του μοντέλου του.
	}
	
	\item 
	{
		\textbf{Διάγνωση Βλαβών.} Η κεντρική ιδέα της χρήσης αναλυτικών μοντέλων στην διάγνωση βλαβών είναι η εξής: Κατασκευάζουμε ένα μοντέλο ώστε να περιγράφει το σύστημα στην κατάσταση φυσιολογικής λειτουργίας του. Ομοίως κατασκευάζουμε μοντέλα για κάθε πιθανή βλάβη που μπορεί να υποστεί το σύστημα. Συγκρίνοντας τις εξόδους των μοντέλων με αυτή του πραγματικού συστήματος κατά την διάρκεια λειτουργίας του, μπορούμε να αποφανθούμε εάν πάσχει από κάποια από τις μοντελοποιημένες βλάβες.
	}
	
\end{itemize}


\section{Ιστορική Αναδρομή}
Το πρόβλημα της \textit{online} αναγνώρισης συστημάτων είναι ένα θέμα το οποίο έχει μελετηθεί εκτεταμένα στην βιβλιογραφία του Αυτομάτου Ελέγχου. Οι αρχικές προσπάθειες έγιναν πάνω στα γραμμικά συστήματα, και η συνήθης προσέγγιση ήταν διέγερση του εκάστοτε συστήματος από μια σειρά δοκιμαστικών εισόδων, καταγραφή της απόκρισης του συστήματος και στην συνέχεια προσαρμογή μοντέλων με χρήση διάφορων αλγορίθμων που είχαν αναπτυχθεί. Παρόλο που αρκετές τέτοιες προσπάθειες πέτυχαν κάποια αποτελέσματα, το πρόβλημα αυτής της μεθοδολογίας είναι ότι δεν διεγείρει το σύστημα με κάποιον συστηματικό τρόπο, με αποτέλεσμα σε πολλές περιπτώσεις να μην αναδεικνύεται ολόκληρη η δυναμική του συστήματος.

Περαιτέρω έρευνες μελέτησαν το πρόβλημα της επαρκούς διέγερσης συστημάτων και απέδειξαν πως οι ιδιότητες σύγκλισης και ευρωστίας των αλγορίθμων αναγνώρισης συστημάτων και ελέγχου συνδέονται με την ικανοποίηση μιας συνθήκης που στην βιβλιογραφία ονομάζεται \textit{Συνθήκη Επιμένουσας Διέγερσης}~\cite{green1985persistence}. Ενώ για την περίπτωση των γραμμικών συστημάτων βρέθηκε πώς η συνθήκη της Επιμένουσας Διέγερσης σχετίζεται άμεσα με το συχνοτικό περιεχόμενο των σημάτων διέγερσης, στην περίπτωση της αναγνώρισης μη-γραμμικών συστημάτων δεν έχει σημειωθεί μεγάλη πρόοδος.

%Ίσως να προσθέσω και μια μικρή παράγραφο για Νευρωνικά και Fuzzy εδώ.

Παρακάτω ακολουθεί μια σύντομη ιστορική αναφορά στην πρόοδο που έχει σημειωθεί στον τομέα της αναγνώρισης των μη-γραμμικών συστημάτων, ενώ πιο λεπτομερείς περιγραφές για πολλά από τα αποτελέσματα και τις έννοιες που παρουσιάζονται θα δοθούν στα επόμενα κεφάλαια.

\subsection{Αρχικές Προσπάθειες}
Οι περισσότερες εργασίες που ασχολούνται με το πρόβλημα, μέχρι και σήμερα χρησιμοποιούν ως μοντέλα τα Νευρωνικά Δίκτυα που λόγω των προσεγγιστικών τους ιδιοτήτων είναι ιδανικά για προσέγγιση μη γραμμικών συναρτήσεων. Στην εργασία~\cite{miyamoto1988feedback} οι συγγραφείς χρησιμοποιούν τα νευρωνικά δίκτυα για αναγνώριση και παρακολούθηση τροχιάς ενός ρομποτικού βραχίονα, μια εφαρμογή αναγνώρισης για αυτόματο έλεγχο. Το πρόβλημα εδώ είναι η άγνωστη αντίστροφη δυναμική του βραχίονα που είναι απαραίτητη για την αποσύμπλεξη των εισόδων ελέγχου. Η ιδέα αυτής της εργασίας είναι η χρήση ενός νευρωνικού δικτύου για την προσέγγιση της δυναμικής αυτής και η εκπαίδευση του μέσω της ελαχιστοποίησης των ροπών που παράγονται από έναν PID ελεγκτή κατά τον έλεγχο του συστήματος. Παρόλο που η εφαρμογή δεν απαιτεί την ακριβή εκμάθηση της άγνωστης δυναμικής, η απόδοση του σχήματος ελέγχου βελτιώθηκε σημαντικά μετά από τα πρώτα 30 λεπτά της εκμάθησης.

Άλλες σημαντικές εργασίες είναι αυτές των \textit{Sanner} και \textit{Slotine}~\cite{sanner1992stable} όπου παρουσιάστηκε ένας αναδρομικός αλγόριθμος για την αναγνώριση μη γραμμικών συστημάτων τόσο στην συνεχή όσο και στην διακριτή περίπτωση, καθώς και των \textit{Lu} και \textit{Basar}~\cite{lu1998robust} όπου συγκρίνονται οι αρχιτεκτονικές δικτύων RBF (radial basis functions) και MFN (multilayer feetforward networks) καθώς και διάφοροι αλγόριθμοι εκμάθησης όπως η βαθμωτή κατάβαση (gradient descent), οι γενετικοί αλγόριθμοι καθώς και ο κλασσικός αλγόριθμος back-propagation. Και στις δυο αυτές εργασίες μελετώνται οι επιπτώσεις της συνθήκης της επιμένουσας διέγερσης. Στην εργασία~\cite{sanner1992stable} δίνεται μια συνθήκη ικανοποίησης της επιμένουσας διέγερσης εμπνευσμένη από την περίπτωση των διακριτών συστημάτων, ενώ στην εργασία~\cite{lu1998robust} δείχνεται πως ανάλογα με τον αλγόριθμο και την αρχιτεκτονική αναγνώρισης που χρησιμοποιείται μπορεί η συνθήκη να έχει διαφορετικές προϋποθέσεις που πρέπει να πληρούνται.

\subsection{Επιμένουσα Διέγερση για Μη-Γραμμικά Συστήματα}
Το πρόβλημα της Επιμένουσας Διέγερσης για τις αρχιτεκτονικές δικτύων RBF μελετήθηκε συστηματικά στην εργασία~\cite{kurdila1995persistency}. Συγκεκριμένα, οι \textit{Kurdila}, \textit{Narcowich} και \textit{Ward} απέδειξαν πως όταν το διάνυσμα οπισθοδρομιτών (regressor vector) αποτελείται από ακτινικές συναρτήσεις βάσης (Radial Basis Functions), τότε αρκεί η τροχιά του συστήματος να πληρεί μια συνθήκη εργοδικότητας. Με πιο απλά λόγια, αυτό θα πει πως όταν χρησιμοποιείται ένα νευρωνικό δίκτυο RBF ως το μαθηματικό μοντέλο περιγραφής του συστήματος, για να πληρείται η συνθηκη Επιμένουσας Διέγερσης αρκεί η τροχιά να διέρχεται απ'όλα τα κέντρα των συναρτήσεων βάσης του δικτύου με έναν περιοδικό τρόπο.

Η ακριβής επίπτωση των επιπέδων διέγερσης στην σύγκλιση ενός αλγορίθμου μελετήθηκε για πρώτη φορά το 2011 στην εργασία~\cite{yuan2011persistency}, ενώ στην εργασία~\cite{zheng2017relationship} δόθηκαν οι μαθηματικές σχέσεις που συνδέουν ποσοτικά τα επίπεδα διέγερσης $a_1$ και $a_2$ με την αρχιτεκτονική του εκάστοτε δικτύου RBF καθώς και τις ιδιότητες της περιοδικής τροχιάς του συστήματος κλειστού βρόγχου.

Καθώς η ικανοποίηση της ΣΕΔ είναι ένα πολύ σημαντικό κομμάτι για στην αναγνώριση συστημάτων, ένα μεγάλο κομμάτι των εργασιών που ασχολούνται με το πρόβλημα βασίστηκε στα αποτελέσματα της εργασίας~\cite{kurdila1995persistency}. Στις εργασίες \cite{wang2003deterministic,wang2004learning,wang2006learning,liu2006learning,liu2007learning,liu2009learning,wang2012learning,wang2012learningpure,wang2013identification,dai2014dynamic,wang2014dynamic,wang2015learning} οι συγγραφείς προτείνουν αλγορίθμους αναγνώρισης διαφόρων ειδών συστημάτων όπως τα strict-feedback και pure-feedback, συστήματα σε κανονική μορφή, καθώς και περιπτώσεις συστημάτων με γνωστό κέρδος εισόδου. Όλες οι παραπάνω εργασίες είναι βασισμένες σε ένα κοινό αποτέλεσμα που είναι εμπνευσμένο από την εργασία~\cite{kurdila1995persistency}, το οποίο ονομάζεται στην βιβλιογραφία \textit{Μερική Συνθήκη Επιμένουσας Διέγερσης (Partial Peristancy of Excitation condition)}. Η ιδέα είναι πως στην περίπτωση που έχω μια τροχιά με περιοδικό χαρακτήρα, μπορώ να χρησιμοποιήσω ένα νευρωνικό δίκτυο RBF με ομοιόμορφα κατανεμημένα κέντρα σε ένα σύνολο που να περιλαμβάνει την τροχιά αυτή. Παρόλο που δεν μπορώ να εξασφαλίσω την ΣΕΔ για την δυναμική του συστήματος σε ολόκληρο το σύνολο $\Omega$, μπορώ να εγγυηθώ μερική σύγκλιση του δικτύου για τα κέντρα που βρίσκονται κοντά στην περιοδική τροχιά, επιτυγχάνοντας έτσι τοπική αναγνώριση του άγνωστου συστήματος.

Ενώ η παραπάνω μεθοδολογία είναι βάσιμη και έχει πετύχει κάποια πολύ σημαντικά αποτελέσματα στην βιβλιογραφία, διακρίνονται τα εξής μειονεκτήματα. Αρχικά η αναγνώριση δεν λαμβάνει χώρο σε ολόκληρο το σύνολο ενδιαφέροντος, παρά μόνο γύρω από μια κλειστή τροχιά. Κατά δεύτερον, ακόμα και σε αυτή την περίπτωση, επιτυγχάνεται αναγνώριση της δυναμικής του συστήματος κλειστού βρόγχου, και κατά συνέπεια τα αποτελέσματα είναι χρήσιμα κυρίως για εφαρμογές ελέγχου και είναι δύσκολο να χρησιμοποιηθούν για εφαρμογές πρόβλεψης ή προσομοίωσης.

\subsection{Πρόσφατη Βιβλιογραφία}
Σχετικά με τις πιο πρόσφατες εξελίξεις στον κλάδο της αναγνώρισης μη γραμμικών συστημάτων, η εργασία~\cite{wang2016dynamic} προσπαθεί να αντιμετωπίσει το πρόβλημα της ραγδαίας αύξησης υπολογιστικών απαιτήσεων στα stict feedback συστήματα μεγάλης τάξης. Μέσω ενός προτεινόμενου μετασχηματισμού, το σύστημα μετατρέπεται σε σύστημα κανονικής μορφής (normal form), και στην συνέχεια μέσω της χρήσης ενός παρατηρητή υψηλού κέρδους (High Gain Observer) καθώς και ενός ελεγκτή προδιαγεγραμμένης απόκρισης (Prescribed Performace Controller) πραγματοποιείται εκτίμηση των άγνωστων μετασχηματισμένων καταστάσεων καθώς και παρακολούθηση της επιθυμητής τροχιάς. Το αποτέλεσμα της παραπάνω προσπάθειας είναι η εκμάθηση της δυναμικής κλειστού βρόγχου του μετασχηματισμένου συστήματος, η οποία στην συνέχεια χρησιμοποιείται για αποτελεσματικότερο έλεγχο.

Στην εργασία~\cite{khazaei2017radial}, οι συγγραφείς προτείνουν μια παραλλαγή του αλγορίθμου \textit{Βαθμωτής Κατάβασης (Gradient Descent)} σε συνδιασμό με ένα σχήμα ελέγχου που στην βιβλιογραφία ονομάζεται \textit{Fast Terminal Sliding Mode Control}. Τα επιτεύγματα της εν λόγω εργασίας είναι η βελτίωση του ρυθμού σύγκλισης των παραμέτρων καθώς και η μείωση του μέσου τετραγωνικού σφάλματος ρίζας. Η μέθοδος σχεδιάστηκε για συστήματα με μοναδιαίο κέρδος ελέγχου και τα αποτελέσματα της επιβεβαιώνονται πειραματικά χρησιμοποιώντας ως παράδειγμα μια παραμετροποιημένη εκδοχή του ταλαντωτή \textit{Duffing} (Dufffing Oscillator) έτσι ώστε να παρουσιάζει χαοτική συμπεριφορά.

Στην εργασία~\cite{yang2017pattern} παρουσιάζεται ένα γενικότερο πλαίσιο ελέγχου, βασισμένο στην θεωρία Αναγνώρισης Προτύπων καθώς και στην Αναγνώριση Συστημάτων, το οποίο μιμείται την ανθρώπινη συμπεριφορά. Πιο συγκεκριμένα, η ιδέα εδώ είναι πως χρησιμοποιώντας την θεωρία αναγνώρισης δυναμικών συστημάτων που παρουσιάστηκε στις προηγούμενες παραγράφους, μπορεί κανείς να εκπαιδεύσει νευρωνικά δίκτυα για διάφορες καταστάσεις ελέγχου, φυσιολογικές ή μη. Στην συνέχεια μέσω της αναγνώρισης προτύπων, ο ελεγκτής είναι ικανός να αναγνωρίσει σε ποια κατάσταση βρίσκεται το σύστημα, και χρησιμοποιώντας την καταχωρημένη γνώση να το ελέγξει πολύ αποδοτικά. Έτσι λοιπόν η προτεινόμενη αρχιτεκτονική μιμείται την ανθρώπινη συμπεριφορά με την έννοια πως είναι ικανή να μαθαίνει και να εκτελεί αποδοτικά σύνθετους στόχους ελέγχου, καθώς και να αναγνωρίζει την κατάσταση στην οποία βρίσκεται, όπως ο άνθρωπος. 

Τέλος, εκτός από τις εξελίξεις πάνω στην θεωρητική θεμελίωση της Αναγνώρισης Συστημάτων, υπάρχουν πάρα πολλές εργασίες με εφαρμογές πάνω σε σχεδόν οποιοδήποτε κλάδο της επιστήμης και της μηχανικής. Κάποιες από τις τελευταίες περιλαμβάνουν αναγνώριση ρομποτικών βραχιόνων~\cite{wang2017dynamic}, εκμάθηση δυναμικής θαλάσσιων σκαφών με σκοπό την επίτευξη αυστηρών στόχων ελέγχου~\cite{dai2016neural}, και ακόμα και ποιοτική σύγκριση δυναμικών συστημάτων~\cite{dong2016modeling}.


\section{Δομή της Διπλωματικής Εργασίας}
Το υπόλοιπο κομμάτι της διπλωματικής εργασίας αποτελείται από 5 ενότητες, που αντιστοιχούν με τα Κεφάλαια 2-6, με την τρέχουσα εισαγωγή να αποτελεί το Κεφάλαιο 1.

Στο Κεφάλαιο 2 γίνεται μια εισαγωγή στις έννοιες και στα μαθηματικά εργαλεία που θα χρησιμοποιηθούν σε όλη την έκταση αυτής της εργασίας. 

Στο Κεφάλαιο 3 παρουσιάζεται αναλυτικά η μέθοδος αναγνώρισης που αναπτύσσεται στα πλαίσια της εργασίας. Αρχικά παρουσιάζονται τα βήματα σχεδίασης που απαιτούνται για την εφαρμογή της μεθόδου σε ένα πραγματικό πρόβλημα, και στην συνέχεια αναλύεται μέσω μαθηματικών επιχειρημάτων η ορθότητα λειτουργίας της.

Στο Κεφάλαιο 4 ακολουθούν κάποια παραδείγματα εφαρμογής της μεθόδου, τόσο σε πραγματικά όσο και σε τεχνητά συστήματα. Μέσω των παραδειγμάτων αυτών γίνεται εμφανής η ικανότητα του σχήματος να αναγνωρίσει την δυναμική άγνωστων συστημάτων, αλλά προκύπτουν και κάποια χρήσιμα συμπεράσματα για την λειτουργία καθώς και τα όρια του.

Στο Κεφάλαιο 5 παρουσιάζουμε κάποιες πιθανές επεκτάσεις που ενδέχεται να βελτιώσουν την ποιότητα των αποτελεσμάτων αναγνώρισης που επιφέρει η μέθοδος. Στην συνέχεια παρουσιάζεται ένας τρόπος αξιολόγησης των αποτελεσμάτων για περιπτώσεις πραγματικών συστημάτων στις οποίες είναι αδύνατη η σύγκριση των αποτελεσμάτων με τις εσωτερικές μη γραμμικές συναρτήσεις του συστήματος.

Στο Κεφάλαιο 8 θα παρουσιαστεί η βιβλιογραφία που χρησιμοποιήθηκε κατά τη διάρ-
κεια εκπόνησης της διπλωματικής εργασίας, και στην οποία θα γίνονται αναφορές σε
διάφορα σημεία του παρόντος εγγράφου.

