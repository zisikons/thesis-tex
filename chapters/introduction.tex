\chapter{Εισαγωγή}

Από την πρώτη της εμφάνισης του πάνω στη γη, ο άνθρωπος κυριαρχείται από μια εναγώνια προσπάθεια κατάκτησης γνώσεων. Ορίζοντας ώς σύστημα κάθε αντικείμενο ή ομάδα αντικειμένων τις ιδιότητες τον οποίων θέλουμε να μελετήσουμε, με τον όρο \textit{αναγνώριση συστημάτων} αναφερόμαστε στην διαδικασία εξαγωγής ενός μαθηματικού μοντέλου του πραγματικού συστήματος με βάση πειραματικά δεδομένα. 

%\textbf{Η εισαγωγική παράγραφος μπορεί να αλλάξει} \\
Το πρόβλημα της αναγνώρισης συστημάτων απασχολεί την επιστημονική κοινότητα για πάνω από μισό αιώνα. Το βασικό κίνητρο είναι πως ένα "καλό" μοντέλο του πραγματικού συστήματος είναι απαραίτητο για μια πληθώρα εφαρμογών. Έτσι λοιπόν ο σχεδιασμός κατάλληλων πειραμάτων, η επιλογή μαθηματικών μοντέλων καθώς και η ανάπτυξη αλγορίθμων εκτίμησης παραμέτρων αποτελούν  μέχρι και σήμερα πεδίο διαρκούς έρευνας.

\section{Εισαγωγικές Έννοιες}
Σκοπός αυτού του κεφαλαίου είναι η παρουσίαση των βασικών εννοιών της θεωρίας \textit{Αναγνώρισης Συστημάτων}. Με αυτό τον τρόπο ελπίζουμε αφενός να γίνει πλήρως κατανοητός ο σκοπός της παρούσης εργασίας και αφετέρου να αποσαφηνισθoύν οι διαφορές με άλλες κλασσικές μεθόδους αναγνώρισης συστημάτων.

\subsection{Μαθηματικά Μοντέλα}
Όπως είπαμε, το αποτέλεσμα της αναγνώρισης συστημάτων στα πλαίσια που την μελετάμε ονομάζεται \textit{μοντέλο}. Υπάρχουν πολλές κατηγορίες μοντέλων όπως τα λεκτικά, μαθηματικά, φυσικά και άλλα, ωστόσο στα πλαίσια αυτής της εργασίας θα εργαστούμε με τα μαθηματικά μοντέλα συστημάτων.

Στην περίπτωση μας λοιπόν, ένα μοντέλο είναι μια μαθηματική σχέση μεταξύ μεταβλητών εισόδου και εξόδου που περιέχει ελεύθερες παραμέτρους. Παραδείγματα μαθηματικών μοντέλων αποτελούν οι συναρτήσεις μεταφοράς με μεταβλητά μηδενικά και πόλους, οι εξισώσεις κατάστασης με άγνωστους πίνακες καταστάσεων καθώς και οι παραμετροποιημένες μη-γραμμικές συναρτήσεις.

Για παράδειγμα, η παρακάτω διαφορική εξίσωση αποτελεί ένα απλό μαθηματικό μοντέλο:
\begin{equation*}
	\dot{x}(t) = -a x(t) + b u(t)
\end{equation*}
όπου οι μεταβλητές $a$ και $b$ είναι οι ελευθέροι παράμετροι του μοντέλου. Η πολυπλοκότητα του επιλεγμένου μοντέλου θα πρέπει να εξυπηρετεί της απαιτήσεις της εκάστοτε εφαρμογής αναγνώρισης.

Λόγω της πληθώρας και της ιδιαιτερότητας των συστημάτων, έχουν προταθεί διάφορα μαθηματικά μοντέλα με χαρακτηριστικά που εξαρτώνται από τις ιδιότητες του προς μελέτη συστήματος. Παρακάτω αναφέρουμε κάποιες βασικές υποκατηγορίες μαθηματικών μοντέλων.

\begin{itemize}
	\item{ 
	\textbf{Ντετερμινιστικά - Στοχαστικά.} Ένα μοντέλο ονομάζεται ντετερμινιστικό, αν περιγράφεται από μια πλήρως προσδιορισμένη σχέση μεταξύ των μεταβλητών του. Αντιθέτως θα λέγεται στοχαστικό, αν εκφράζεται μέσω πιθανοθεωρίας.
	}
	
	\item{ 
	\textbf{Στατικά - Δυναμικά.} Εάν η σχέση (μοντέλο) που συνδέει τις μεταβλητές ενός συστήματος δεν εξαρτάται από παρελθοντικές τιμές των μεταβλητών θα λέμε ότι το σύστημα, άρα και το μοντέλο, είναι στατικό. Στην αντίθετη περίπτωση θα λέγεται δυναμικό. Ένα παράδειγμα στατικού συστήματος είναι αυτό που περιγράφεται από αλγεβρικές εξισώσεις, ενώ συνήθως τα δυναμικά συστήματα - μοντέλα περιγράφονται από διαφορικές εξισώσεις ή εξισώσεις διαφορών.
	}
	
	\item{ 
	\textbf{Συνεχούς - Διακριτού Χρόνου.} Η ιδιότητα αυτή ορίζει τον τρόπο με τον οποίο η μεταβλητή του χρόνου επιδρά στο μοντέλο. Διακριτού χρόνου ονομάζονται τα μοντέλα τα οποία εκφράζουν την σχέση που συνδέει τις μεταβλητές του συστήματος σε διακριτές χρονικές στιγμές. Eν αντιθέσει, τα συστήματα στα οποία ο χρόνος είναι συνεχής μεταβλητή ονομάζονται συστήματα συνεχούς χρόνου. Τα μοντέλα διακριτού χρόνου περιγράφονται συνήθως από εξισώσεις διαφορών ενώ τα συστήματα συνεχούς χρόνου από διαφορικές εξισώσεις.
	}
	
	\item{ 
	\textbf{Χρονομεταβλητά - Χρονοαμετάβλητα.} Χρονομεταβλητά ονομάζονται τα συστήματα (και τα μοντέλα συστημάτων) στα οποία οι εξισώσεις που περιγράφουν την λειτουργιά του συστήματος έχουν άμεση εξάρτηση από τον χρόνο. Κατά συνέπεια, σε αυτά τα συστήματα είναι πιθανό η ίδια είσοδος σε διαφορετικές χρονικές στιγμές να οδηγήσει σε διαφορετική απόκριση του συστήματος. Έν αντιθέσει, χρονοαμετάβλητα είναι τα συστήματα στα οποία η εξάρτηση με τον χρόνο εκφράζεται μόνο έμμεσα μέσω των εσωτερικών καταστάσεων ή της συνάρτησης εισόδου του συστήματος. 
	
	
%	Χρονομεταβλητά ονομάζονται τα συστήματα (και τα μοντέλα συστημάτων) στα οποία ο χρόνος έχει άμεση επίδραση. Με άλλα λόγια είναι συστήματα στα οποία κάποιες συγκεκριμένες ποσότητες που επηρεάζουν την συμπεριφορά του συστήματος έχουν άμεση συσχέτιση με τον χρόνο, και ως εκ τούτου η ίδια είσοδος σε διαφορετικές χρονικές στιγμές ενδέχεται να οδηγήσει σε διαφορετικές αποκρίσεις. Αντίθετα στα χρονοαμετάβλητα συστήματα, οι εξισώσεις που περιγράφουν την λειτουργιά του συστήματος δεν έχουν άμεση εξάρτηση από τον χρόνο.
	
	}
\end{itemize}

Τα συστήματα που θα μελετήσουμε σε αυτή την εργασία περιγράφονται από μη-γραμμικές διαφορικές εξισώσεις όπου οι μη-γραμμικές συναρτήσεις που διέπουν την λειτουργία τους εξαρτώνται μόνο από τις καταστάσεις και την είσοδο ελέγχου. Με βάση την παραπάνω κατηγοριοποίηση λοιπόν, τα συστήματα (και τα αντίστοιχα μοντέλα) που μελετάμε είναι μη-γραμμικά, χρονοαμετάβλητα δυναμικά συστήματα συνεχούς χρόνου.

\pagebreak
\subsection{Εκτίμηση Παραμέτρων}
Στην αναγνώριση συστημάτων, αφού επιλέξουμε ένα μαθηματικό μοντέλο με την ικανότητα να περιγράψει επαρκώς την λειτουργία του συστήματος, το επόμενο στάδιο είναι ο προσδιορισμός των ελεύθερων του παραμέτρων. Η διαδικασία αυτή στην βιβλιογραφία ονομάζεται \textit{εκτίμηση παραμέτρων} (parameter estimation).

%H διαδικασία αυτή περιλαμβάνει τον σχεδιασμό ενός πειράματος με σκοπό την συλλογή δεδομένων για το σύστημα που μελετάται, και στην συνέχεια την χρήση τους από κάποιο \textit{αλγόριθμο εκτίμησης παραμέτρων} με σκοπό 

Σε αυτό το στάδιο, είναι απαραίτητος ο σχεδιασμός ενός πειράματος με σκοπό την συλλογή δεδομένων για το σύστημα που μελετάται. Στην συνέχεια, τα δεδομένα χρησιμοποιούνται από κάποιον αλγόριθμο με σκοπό τον προσδιορισμό των ελεύθερων παραμέτρων του μοντέλου που έχει επιλεχθεί. Η αποτελεσματικότητα του αλγορίθμου εκτίμησης παραμέτρων είναι άμεση συνάρτηση των δεδομένων που θα χρησιμοποιηθούν, συνεπώς απαιτείται προσοχή κατά τον σχεδιασμό του πειράματος συλλογής δεδομένων.

Διακρίνονται δύο μεγάλες οικογένειες αλγορίθμων εκτίμησης παραμέτρων:

\begin{itemize}
	\item{\textbf{Offline:} Οι \textit{offline} αλγόριθμοι απαιτούν την εκ των προτέρων συλλογή δεδομένων για διαθέσιμο σύστημα. Στην συνέχεια τα δεδομένα αυτά επεξεργάζονται από κάποιον αλγόριθμο εκτίμησης παραμέτρων με σκοπό την προσαρμογή ενός υποψηφίου μοντέλου του συστήματος. Το μεγάλο πλεονέκτημα των \textit{pffline} αλγορίθμων είναι το γεγονός ότι δεν υπάρχει φραγμός ως προς την υπολογιστική τους πολυπλοκότητα, συνεπώς μπορούν να χρησιμοποιηθούν σύνθετοι αλγόριθμοι βελτιστοποίησης που καθιστούν εφικτή την προσαρμογή ακόμα και πολύ σύνθετων μοντέλων όπως τα \textit{πολυεπίπεδα νευρωνικά δίκτυα} (Deep Neural Networks). Ένα κλασσικό παράδειγμα τέτοιου αλγορίθμου είναι η \textit{Μέθοδος των Ελαχίστων Τετραγώνων}.
	}
		
	\item{\textbf{Online:} Αντίθετα με τους \textit{offline} αλγορίθμους οι \textit{online} αλγόριθμοι πραγματοποιούν εκτίμηση παραμέτρων σε πραγματικό χρόνο χρησιμοποιώντας τα δεδομένα που συλλέγονται κατά την διάρκεια λειτουργίας του πραγματικού συστήματος. Το γεγονός αυτό επιβάλει σε αυτούς τους αλγορίθμους να είναι υπολογιστικά απλοί, καθώς οι υπολογισμοί δεν μπορούν να διαρκούν περισσότερο από τον κύκλο λειτουργίας του συστήματος. 
		
	Ωστόσο, το πλεονέκτημα που προσφέρουν είναι ότι το μοντέλο είναι διαθέσιμο κατά την διάρκεια λειτουργίας του συστήματος, ιδιότητα που τους καθιστά ιδιαίτερα χρήσιμους σε εφαρμογές όπως ο \textit{προσαρμοστικός έλεγχος} (adaptive control) και η \textit{διάγνωση βλαβών} (fault detection). Ένας κλασσικός \textit{online} αλγόριθμος εκτίμησης παραμέτρων είναι η \textit{Αναδρομική Μέθοδος Ελαχίστων Τετραγώνων}.
	}
\end{itemize}

Για τους σκοπούς της παρούσας εργασίας η μέθοδος εκτίμησης που θα αναπτυχθεί πρόκειται για μια \textit{online} μέθοδο.


\section{Εφαρμογές της Αναγνώρισης Συστημάτων}

\section{Διαδικασία Αναγνώρισης Συστημάτων}

\section{Σημαντική Βιβλιογραφία}

\section{Δομή της Διπλωματικής Εργασίας}

\subsection{(τίτλος υποενότητας 1.1.1)}


