\chapter{Εισαγωγή}

Από την πρώτη της εμφάνισης του πάνω στη γη, ο άνθρωπος κυριαρχείται από μια εναγώνια προσπάθεια κατάκτησης γνώσεων. Ορίζοντας ώς σύστημα κάθε αντικείμενο ή ομάδα αντικειμένων τις ιδιότητες τον οποίων θέλουμε να μελετήσουμε, με τον όρο \textit{αναγνώριση συστημάτων} αναφερόμαστε στην διαδικασία εξαγωγής ενός μαθηματικού μοντέλου του πραγματικού συστήματος με βάση πειραματικά δεδομένα. 

\textbf{Η εισαγωγική παράγραφος μπορεί να αλλάξει} \\
Το πρόβλημα της αναγνώρισης συστημάτων απασχολεί την επιστημονική κοινότητα για πάνω από μισό αιώνα. Το βασικό κίνητρο είναι πως ένα "καλό" μοντέλο του πραγματικού συστήματος είναι απαραίτητο για μια πληθώρα εφαρμογών %όπως αυτές του αυτομάτου ελέγχου, της προσομοίωσης, της πρόβλεψης, της εκτίμησης καταστάσεων καθώς και της ανίχνευσης σφαλμάτων. 
Έτσι λοιπόν ο σχεδιασμός κατάλληλων πειραμάτων, η επιλογή μαθηματικών μοντέλων καθώς και η ανάπτυξη αλγορίθμων εκτίμησης παραμέτρων αποτελούν  μέχρι και σήμερα πεδίο διαρκούς έρευνας.

\section{Εισαγωγικές Έννοιες}
Σκοπός αυτού του κεφαλαίου είναι η παρουσίαση των βασικών εννοιών της θεωρίας \textit{Αναγνώρισης Συστημάτων}. Με αυτό τον τρόπο ελπίζουμε αφενός να γίνει πλήρως κατανοητός ο σκοπός της παρούσης εργασίας και αφετέρου να αποσαφηνισθoύν οι διαφορές με άλλες κλασσικές μεθόδους αναγνώρισης συστημάτων.

\subsection{Κατηγορίες Μαθηματικών Μοντέλων}
Όπως είπαμε, το αποτέλεσμα της αναγνώρισης συστημάτων στα πλαίσια που την μελετάμε ονομάζεται \textit{μοντέλο}. Ένα μοντέλο 

\subsection{Αλγόριθμοι εκτίμησης παραμέτρων}


\section{Εφαρμογές της Αναγνώρισης Συστημάτων}

\section{Διαδικασία Αναγνώρισης Συστημάτων}

\section{Σημαντική Βιβλιογραφία}

\section{Δομή της Διπλωματικής Εργασίας}

\subsection{(τίτλος υποενότητας 1.1.1)}


